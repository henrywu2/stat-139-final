\documentclass[11pt]{article}
\usepackage[margin=1in]{geometry}
\usepackage{parskip}
\usepackage{hyperref}

\begin{document}

\begin{center}
    \Large\textbf{Socioeconomic Determinants of County-Level Voter Turnout (2000-2020)}

    \vspace{0.3cm}
    \normalsize
    Henry Wu \quad Yuen Ler Chow \quad John Rho

    \vspace{0.2cm}
    \small
    November 15, 2024
\end{center}

\section*{Variables and Data Sources}
We utilize demographic and economic data from Opportunity Insights combined with county-level voter turnout data (2000-2020) to identify key demographic predictors of voter turnout and analyze cross-county trends. Our voter turnout calculations use presidential election returns from the MIT Election Data and Science Lab\footnote{MIT Election Data and Science Lab, 2018, "County Presidential Election Returns 2000-2020", \url{https://doi.org/10.7910/DVN/VOQCHQ}, Harvard Dataverse, V13} combined with Census voting-age population estimates. Key variables include:
\begin{itemize}\setlength{\itemsep}{0pt}
    \item Dependent Variable: Voter turnout (voters/voting age population)
    \item Economic Indicators: Household income, poverty rates, job growth
    \item Demographics: Educational attainment, racial composition, foreign-born population
    \item Geographic Factors: Population density, commute times
    \item Social Indicators: Single-parent households, mail return rates
\end{itemize}

\section*{Planned Analyses}
\begin{itemize}\setlength{\itemsep}{0pt}
    \item Research Objectives
          \begin{itemize}\setlength{\itemsep}{0pt}
              \item Identify which demographic factors are the strongest predictors of voter turnout
              \item Analyze cross-county trends and patterns in voter participation
              \item Quantify the marginal effects of each socioeconomic factor on turnout
          \end{itemize}
    \item Exploratory Data Analysis
          \begin{itemize}\setlength{\itemsep}{0pt}
              \item Test for multicollinearity using VIF
              \item Examine temporal trends and spatial patterns
              \item Assess distributional properties of variables
          \end{itemize}
    \item Model Development
          \begin{itemize}\setlength{\itemsep}{0pt}
              \item Primary: Beta regression (suitable for [0,1] bounded dependent variable)
              \item Alternative: Fractional logistic regression
              \item Panel data methods to leverage temporal dimension
              \item Calculation of average marginal effects for key predictors
              \item Cross-validation to assess predictive power of demographic variables
          \end{itemize}
\end{itemize}

\section*{Implications}
Understanding the socioeconomic determinants of voter turnout has important implications for both policy and democratic participation:
\begin{itemize}\setlength{\itemsep}{0pt}
    \item Policy Design
          \begin{itemize}\setlength{\itemsep}{0pt}
              \item Identify barriers to voting faced by specific demographic groups
              \item Guide resource allocation for voter outreach and education programs
              \item Inform placement of polling locations and early voting sites
          \end{itemize}
    \item Democratic Engagement
          \begin{itemize}\setlength{\itemsep}{0pt}
              \item Better understand patterns of political representation across communities
              \item Develop targeted strategies to increase civic participation
              \item Address systemic inequalities in electoral participation
          \end{itemize}
    \item Research Contribution
          \begin{itemize}\setlength{\itemsep}{0pt}
              \item Provide updated evidence on turnout determinants using recent data
              \item Establish causal relationships between socioeconomic factors and voting behavior
              \item Create predictive models for identifying low-turnout communities
          \end{itemize}
\end{itemize}

\section*{Challenges}
\begin{itemize}\setlength{\itemsep}{0pt}
    \item Calculating accurate voter turnout denominators required extensive processing of Census data to determine voting-age population
    \item Temporal alignment of various data sources spanning different years
    \item Managing potential spatial autocorrelation in county-level data
    \item Handling missing data and ensuring consistent geographic definitions over time
\end{itemize}

\section*{Data Dictionary}
Key variables from Opportunity Insights\footnote{Chetty, R., Friedman, J. N., Hendren, N., Jones, M. R., \& Porter, S. R. (2018). The Opportunity Atlas: Mapping the Childhood Roots of Social Mobility [Data set]. Opportunity Insights. https://opportunityinsights.org/data/}:
\begin{itemize}\setlength{\itemsep}{0pt}
    \item Educational: College degree attainment (frac\_coll\_plus2010), 3rd grade math scores (gsmn\_math\_g3\_2013)
    \item Demographics: Foreign-born share, racial composition (2010)
    \item Economic: Median household income (2016), poverty rates, employment, wage growth
    \item Housing: Two-bedroom rent (2015)
    \item Geographic: Population density, job density (2013)
    \item Social: Single parent households, commute times (2010)
\end{itemize}

\end{document}
